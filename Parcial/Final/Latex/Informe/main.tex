\documentclass[osajnl,twocolumn,showpacs,superscriptaddress,10pt]{revtex4-2}
%
%
\usepackage{dcolumn}% Align table columns on decimal point
\usepackage{bm}
\usepackage{bbm}
\usepackage{lipsum}
\usepackage[spanish,es-tabla]{babel}
\usepackage[utf8]{inputenc}
\usepackage[T1]{fontenc}
\usepackage{makeidx}
\usepackage{graphicx}
\usepackage{animate}
\usepackage{subfig}
\usepackage{gensymb}
\usepackage{physics}
\usepackage{amsmath}
\usepackage{amsfonts}
\usepackage{amssymb}
\usepackage{mathrsfs}
\usepackage[pdftex]{hyperref}
\usepackage{multirow}
\usepackage{float}
\usepackage{booktabs}
\usepackage{wrapfig}
\usepackage{minipage-marginpar}
\usepackage{boxedminipage}
\usepackage[dvipsnames]{xcolor}
%
\usepackage{fancyhdr}
\usepackage{fourier-orns}
%
\renewcommand{\headrule}{%
\vspace{2pt}\hrulefill
\raisebox{0pt}{\quad\decofourleft\decotwo\decofourright\quad}%
\hrulefill}
\setlength{\headheight}{35pt}
%
%
\AtBeginDocument{\selectlanguage{spanish}}
\decimalpoint
%\bibliographystyle{IEEEtran}
%\bibliography{IEEEabrv,mybibfile}
%
%
\begin{document}
%Titulo
%
%Comandos--------------------------------
\providecommand{\hs}[1]{\ensuremath{\hspace{#1 pt}}}
\providecommand{\hsn}[1]{\ensuremath{\hspace{#1 pt}}}
\providecommand{\vs}[1]{\ensuremath{\vspace{#1 pt}}}
\providecommand{\unitvec}[1]{\boldsymbol{\hat{#1}}}
\providecommand{\dif}[2]{ \ensuremath{\frac{d #1}{d #2}} }
\providecommand{\into}[2]{ \ensuremath{\oint_{#1}{#2}} }

\providecommand{\linft}[2]{ \ensuremath{\lim_{#1\rightarrow\infty}\corch{#2}} }

\providecommand{\parent}[1]{\ensuremath{\left(\hspace{2pt} #1 \hspace{2pt}\right)}}
\providecommand{\corch}[1]{\ensuremath{\left[\hspace{2pt} #1 \hspace{2pt}\right]}}
\providecommand{\llav}[1]{\ensuremath{\left\{\hspace{2pt} #1 \hspace{2pt}\right\}}}

\providecommand{\difn}[3]{ \ensuremath{\frac{d^{#3} #1}{d {#2}^{#3}}} }
\providecommand{\parti}[2]{ \ensuremath{\frac{\partial #1}{\partial #2}} }
\providecommand{\partin}[3]{ \ensuremath{\frac{{\partial}^{#3} #1}{{\partial} {#2}^{#3}}} }
\providecommand{\bs}[1]{\ensuremath{\boldsymbol{#1}}}

\providecommand{\pcruz}[2]{\ensuremath{ {#1}\times{#2} }}
\providecommand{\ppunt}[2]{\ensuremath{ {#1}\cdot{#2} }}
\providecommand{\bs}[1]{\ensuremath{ \boldsymbol{#1} }}

\providecommand{\llint}[4]{\ensuremath{\int_{#1}^{#2}\left\{\hsn{2} #3 \hsn{2}\right\}\hsn{1}{d{#4}} }}
\providecommand{\chint}[4]{\ensuremath{\int_{#1}^{#2}\left[\hsn{2} #3 \hsn{2}\right]\hsn{1}{d{#4}} }}
\providecommand{\print}[4]{\ensuremath{\int_{#1}^{#2}\left(\hsn{2} #3 \hsn{2}\right)\hsn{1}{d{#4}}}}
\providecommand{\clint}[3]{\ensuremath{\left[\hs{2} {#3} \hs{2}\right|_{#1}^{#2} }}
\providecommand{\inte}[4]{\ensuremath{ \int_{#1}^{#2}{#3}\hsn{1}{d{#4}} }}
\providecommand{\lne}[1]{\ensuremath{\ln{\left\{\hspace{1pt} #1 \hspace{1pt}\right\}}}}

\providecommand{\cose}[1]{\ensuremath{\cos{\left[\hspace{2pt} #1 \hspace{2pt}\right]}}}
\providecommand{\seno}[1]{\ensuremath{\sin{\left[\hspace{2pt} #1 \hspace{2pt}\right]}}}
\providecommand{\tang}[1]{\ensuremath{\tan{\left[\hspace{2pt} #1 \hspace{2pt}\right]}}}
\providecommand{\cota}[1]{\ensuremath{\cot{\left[\hspace{2pt} #1 \hspace{2pt}\right]}}}
\providecommand{\seca}[1]{\ensuremath{\sec{\left[\hspace{2pt} #1 \hspace{2pt}\right]}}}
\providecommand{\csec}[1]{\ensuremath{\csc{\left[\hspace{2pt} #1 \hspace{2pt}\right]}}}

\providecommand{\cosen}[2]{\ensuremath{\cos^{#2}{\left[\hspace{2pt} #1 \hspace{2pt}\right]}}}
\providecommand{\senon}[2]{\ensuremath{\sin^{#2}{\left[\hspace{2pt} #1 \hspace{2pt}\right]}}}
\providecommand{\tangn}[2]{\ensuremath{\tan^{#2}{\left[\hspace{2pt} #1 \hspace{2pt}\right]}}}
\providecommand{\cotan}[2]{\ensuremath{\cot^{#2}{\left[\hspace{2pt} #1 \hspace{2pt}\right]}}}
\providecommand{\secan}[2]{\ensuremath{\sec^{#2}{\left[\hspace{2pt} #1 \hspace{2pt}\right]}}}
\providecommand{\csecn}[2]{\ensuremath{\csc^{#2}{\left[\hspace{2pt} #1 \hspace{2pt}\right]}}}

\providecommand{\acose}[1]{\ensuremath{\cos^{-1}{\left[\hspace{2pt} #1 \hspace{2pt}\right]}}}
\providecommand{\aseno}[1]{\ensuremath{\sin^{-1}{\left[\hspace{2pt} #1 \hspace{2pt}\right]}}}
\providecommand{\atang}[1]{\ensuremath{\tan^{-1}{\left[\hspace{2pt} #1 \hspace{2pt}\right]}}}
\providecommand{\acota}[1]{\ensuremath{\cot^{-1}{\left[\hspace{2pt} #1 \hspace{2pt}\right]}}}
\providecommand{\aseca}[1]{\ensuremath{\sec^{-1}{\left[\hspace{2pt} #1 \hspace{2pt}\right]}}}
\providecommand{\acsec}[1]{\ensuremath{\csc^{-1}{\left[\hspace{2pt} #1 \hspace{2pt}\right]}}}

\providecommand{\sumat}[3]{\ensuremath{\sum_{#1}^{#2}{\left\{\hspace{2pt} #3 \hspace{2pt}\right\}}}}
\providecommand{\suma}[3]{\ensuremath{\sum_{#1}^{#2}{#3}}}
\providecommand{\Rarr}[1]{\ensuremath{\hs{#1}\Longrightarrow\hs{#1}}}
\providecommand{\rarr}[1]{\ensuremath{\hs{#1}\longrightarrow\hs{#1}}}

\providecommand{\expo}[1]{\ensuremath{e^{#1}}}
\providecommand{\uvec}[1]{\ensuremath{\hat{#1}}}

\title{La influencia de Genshin Impact en el mercado occidental (improvisado)}
\thanks{Laboratorio de reducción de datos}

\author{Mario Armando, Urbina Silva}\email{mariosva139@gmail.com}
\affiliation{Escuela de Ciencias Físicas y Matemáticas, Universidad de San Carlos de Guatemala, Ciudad Universitaria, Zona 12, Guatemala.
}%
%\affiliation{}
%\collaboration{MUSO Collaboration}%\noaffiliation
%\date{\today}%
%Resumen
\begin{abstract}
    Hola
\end{abstract}
\maketitle
\section{Introducción}
Los videojuegos se han ido introduciendo cada vez más en la sociedad de hoy en día. Al inicio de los años 90, el videojuego se convirtiò en algo que iba adquiriendo forma con el pasar del tiempo, dejó de ser solamente entretenimiento puro, empezó a contar grandes historias, mejores que muchas de las películas del momento, convirtiendo de esta forma a la industria del videojuego en un importante sector económico. Con la entrada de los años 2000 fue el comienzo de la modalidad online de diversos títulos que causaron gran furor en su momento, y a la vez que algo se hace popular, resulta mucho más sencillo sacar probecho de ello.
\\\\
A partir de ello, la industria de los videojuegos se fue consolidando a nivel económico, uniéndose a colosos como la música, el cine y la televisión, y con ello tras la mitad de la década del 2010 dicha industria llegó a facturar más dinero que el cine y la música juntos\cite{ONTSI}. Entre muchas cosas negativas, de esta industría, se puede mencionar que con la creación de los videojuegos surgió la adicción a estos y el tiempo que los jugadores dedicaban a ellos. Dicho problema se ha ido agravando con el paso de los años, y no solamente con los juegos en sí; sino que ha empeorado con la popularización de los juegos gacha, juegos gratuitos con micro transacciones que han estado en el punto de mira por su similitud a las máquinas trapaperras y demás elementos encontrados en casinos, con la diferencia de que los juegos gacha son totalmente gratuitos\cite{Navarro}.
\\\\
Con el avance de la tecnología se tiene hoy en día que cualquier aficionado a los videojuegos pueda encontrar su estilo de juego preferido y disfrutar dicho medio de entretenimiento durante horas. Gracias a las plataformas móviles, los denominados juegos \textit{gacha}, género que en el pasado estuvo limitado a una pequeña base de fans, están cobrando cada vez mas popularidad a nivel mundial.
\\\\
Los juegos gacha son videojuegos que animan a los jugadores a utilizar su propia modena interna para comprar artículos virtuales mientras progresan en la partida. Estos juegos permiten ganar artículos como cartas y personajes a través de un mecanismo muy similar al de las maquinas expendedoras girando una ruleta o realizando una tirada de datos mediante la moneda del juego para obtener un artículo aleatorio. \cite{gacha}
\\\\
Los juegos gacha se han convertido en un gran modelo de negocios en los últimos años, estos empezaron ganando popularidad en Asia, y ahora con la opción portatil, siguen sumando fanáticos de los juegos para móviles en todo el mundo. Este género ha ganado reputación sobre la base de la incertidumbre, esto se debe a que con ofertas de recompensas sorpresa y sus tramas en constante evolución, los videojuegos gacha han estado cautivando a jugadores de todas las tendencias desde 2010. \cite{gacha02}

\section{Definición de un videojuego gacha}
El término gacha proviene de las máquinas dispensadoras japonesas denominadas Gashapón. Su nombre viene de una combinación de diferentes onomatopeyas, la primera parte definiría el sonido de la máquina dispensadora (gasha-gasha) y la segunda parte sería el sonido del objeto saliendo de la máquina (pon). El funcionamiento del gacha es muy sencillo. En primer lugar, se introduce en dinero en la máquina, se gira una manivela o se tira de una palanca y se obtiene una bola de plástico que contiene el objeto que nos ha tocado. Al acto de pagar para obtener un objeto de este tipo de máquinas se le denomina pullear o rollear, que traducido significa tirar y rodar respectivamente, ya que representan la acción de tirar de la palanca (pull) o la acción de hacer girar el mecanismo (roll). Estos términos se utilizan tanto en inglés como en castellano.
\\\\
Al momento de introducir dicho concepto en el mundo de los videojuegos, los desarrolladores vieron en dicho sistema de recompensa un modelo de negocio, especialmente para los videojuegos móviles gratuitos, de manera que desde el 2012 dicho sistema empezó a popularizarse. En general, los videojuegos gacha no se centran en su jugabilidad, la cual suele ser bastante simple al principio y poco a poco se va complicando para mantener a sus jugadores activos, entretenidos y conseguir atraer nuevos usuarios, sino en la obtención de recursos, ya sean personajes, armas, trajes, equipamiento y demás. Para obtener dichos elementos normalmente se realiza una tirada a una máquina sin límite siempre y cuando se tenga la moneda que pide el juego para hacerlo. Estas imitaciones de las máquinas reales han de establecer unas probabilidades para obtener objetos, de forma que, a mayor rareza, menor probabilidad habrá de obtener el objeto.\cite{Navarro}
\\\\
El método de pago es relativamente sencillo, se canjea la moneda del juego por una tirada, y mediante una animación se revela la rareza y el objeto obtenido. Una vez finalizada la tirada de gacha, ya se podrá usar el objeto obtenido sin limitación alguna. Los juegos gacha contienen una gran variedad de características las cuales se pueden dividir de la siguiente manera
\begin{table}[H]
    \begin{center}
        \begin{tabular}{|ccc|} \hline
            \multicolumn{3}{|c|}{Juegos Gacha}                            \\ \hline
            \multicolumn{1}{|c|}{\textbf{Tipos}} & \multicolumn{1}{c|}{\textbf{Géneros}} & \textbf{Mecánicas relacionadas} \\ \hline\hline
            \multicolumn{1}{|c|}{\begin{tabular}[c]{@{}c@{}}Gacha\\ completa\end{tabular}} & \multicolumn{1}{c|}{\begin{tabular}[c]{@{}c@{}}Juegos de\\ estrategia\\ por turnos\end{tabular}} &  \\ \hline
            \multicolumn{1}{|c|}{\begin{tabular}[c]{@{}c@{}}Caja\\Gacha\end{tabular}} & \multicolumn{1}{c|}{\begin{tabular}[c]{@{}c@{}}Juegos\\ inactivos o\\ IDLE\end{tabular}} & Banners \\ \hline
            \multicolumn{1}{|c|}{\begin{tabular}[c]{@{}c@{}}Gacha\\ sugoroku\end{tabular}} & \multicolumn{1}{c|}{Juegos rítmicos} & Objetos de preferencia \\ \hline
            \multicolumn{1}{|c|}{\begin{tabular}[c]{@{}c@{}}Gacha con\\ reintento\end{tabular}} & \multicolumn{1}{c|}{} & Pity Break System \\ \hline
            \multicolumn{1}{|c|}{\begin{tabular}[c]{@{}c@{}}Gacha\\ consecutivo\end{tabular}} & \multicolumn{1}{c|}{} & Spark System \\ \hline
            \multicolumn{1}{|c|}{\begin{tabular}[c]{@{}c@{}}Gacha\\ intensificado\end{tabular}} & \multicolumn{1}{c|}{} & Objetos duplicados \\ \hline
            \multicolumn{1}{|c|}{\begin{tabular}[c]{@{}c@{}}Gacha\\ abierto/cerrado\end{tabular}} & \multicolumn{1}{c|}{} & PvP \\ \hline
            \multicolumn{1}{|c|}{\begin{tabular}[c]{@{}c@{}}Gacha con\\ descuento\end{tabular}} & \multicolumn{1}{c|}{} & Eventos \\ \hline
        \end{tabular}
        \vs{-5}
        \caption{\label{Res-mec-gatcha}Resumen de las mecánicas de los juegos gatcha, Navarro 2021 \cite{Navarro}}
    \end{center}
\end{table}
\section{Genshin Impact}
Genshin Impact es un juego de rol de acción de mundo abierto desarrollado por miHoYo que nos lleva hasta Teyvat, un mundo de fantasía donde los siete elementos fluyen y convergen. En el rol de un viajero de otro mundo recién despertado en esta tierra, el jugador tiene la misión de buscar a Los Arcontes de los Siete Elementos, unos seres mortales. Cada pasaje de la aventura de Genshin Impact desenterrará algo milagroso, y lo perdido será finalmente encontrado.
\\\\
Genshin Impact se hace fuerte en su ambientación, con cierto parecido al vasto mundo de Hyrule creado por Nintendo para The Legend of Zelda: Breath of the Wild pero con suficiente fuerza e identidad propia como para ofrecer al jugador una experiencia visual totalmente inmersiva. En ese objetivo sus responsables garantizan paisajes majestuosos, animaciones desbordantes en tiempo real y movimientos de personajes finamente detallados así como un sistema de iluminación y tiempo que cambia conforme se avanza en la partida, todo rodeado por una cuidada banda sonora.
\\\\
miHoYo también apuesta su capital a un sistema de combate elemental. Su funcionamiento parte de la premisa de dominar los siete elementos de su mundo—Anemo, Electro, Hydro, Pyro, Cryo, Dendro y Geo— para desencadenar diferentes reacciones. Por otra parte, quienes deseen llegar lo más lejos posible en el videojuego se han de armar un poderoso equipo, escogiendo entre una amplia selección de personajes, cada uno con personalidades, historias y habilidades únicas.
\\\\
Genshin Impact es uno de los grandes éxitos free-to-play del 2020 para PC, PS5, PS4 y móviles iOS y Android, y un referente de la nueva hornada de títulos desarrollados en China para Occidente.\cite{3DJuegos}

\section{Metodología}


\section{Resultados}


\section{Conclusiones}
    \begin{itemize}
        \item El rozamiento viscoso se da como consecuencia de la fricción que presenta un fluido respecto a un objeto.
        \item La ley de Poiseuille se cumple fluidos con viscosidad constante, es decir, en fluidos homogéneos.
        \item El número de Reynolds favorece los flujos turbulentos, como consecuencia de la aceleración y desaceleración del pulso, en el caso del flujo sanguíneo.
    \end{itemize}


% \section{Referencias}
%     \begin{itemize}
%         %https://cbccampusvirtual.uba.ar/pluginfile.php/1921518/mod_resource/content/3/Leyes%20OhmPoiseuille.pdf
%         \item Castañaga, A., et al. (2023) Flujo de Poiseuille. [En Línea], Recuperado de: \url{https://mat.caminos.upm.es/wiki/Flujo_de_Poiseuille_(Grupo_23)}
%         \item Ciancaglini, Carlos. (2004). Hidrodinamia de la circulación vascular periférica normal y patológica. Revista Costarricense de Cardiología, 6(2), 43-61. Retrieved May 10, 2024, from \url{http://www.scielo.sa.cr/scielo.php?script=sci_arttext&pid=S1409-41422004000200006&lng=en&tlng=es.}
%         \item Lifshits, E.M. and Landau, L.D. (1986) Mecánica de fluidos (9 vols). 2nd edn. España:  Reverte.
%         \item Galpern, E., Shalom, D., Della, F. (2017) Exp 6: Fuerza viscosa. [PDF], Recuperado de: \url{http://materias.df.uba.ar/f1bygaa2017c1/files/2017/05/Exp6.pdf}
%         \item Olmo, M. (n.d.) Ley de Poiseuille. [En Línea], Recuperado de: \url{http://hyperphysics.phy-astr.gsu.edu/hbasees/poicon.html}
%         %http://laplace.us.es/wiki/index.php/Fuerzas_de_rozamiento_(GIE)
%         \item Pedro. (2023) Fuerzas de rozamiento (GIE), [En Línea]. Recuperado de: \url{http://laplace.us.es/wiki/index.php/Fuerzas_de_rozamiento_(GIE)}
%         %
%         \item Symon, K.R. and Almarza, Y.A. (1979) Mecánica. 1st edn. Madrid: Aguilar.
%         %
%         \item ULPGC. (n.d.) Rozamiento interno: viscosidad, [PDF]. Recuperado de: \url{https://www2.ulpgc.es/hege/almacen/download/43/43822/stokes.pdf}
%         \item Universidad de Cantabria. (2017) Tema 5. Hemodinámica o física del flujo sanguíneo. [En Línea], Recuperado de: \url{https://ocw.unican.es/mod/page/view.php?id=509&lang=en}
%     \end{itemize}


    \bibliography{ref.bib}
\end{document}